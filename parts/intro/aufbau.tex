\section{Aufbau der Arbeit}

Das aktuelle einleitende Kapitel beschreibt die Motivation und Hintergründe, warum die pludoni GmbH die Testgetriebene Entwicklung mit Ruby on Rails\index{Ruby-on-Rails} verwenden möchte und gibt eine kurze Projektbeschreibung über das Projekt IT-Jobs-und-Stellen.de (IT-Jobs)\index{IT-Jobs-Projekt}, welches parallel zu dieser Arbeit entstand.

Da die Testgetriebene\index{TDD} Software-Entwicklung auf dem dynamischen Unittest\index{Test!Unittest} basiert, ist das Kapitel \fullref{sec:test} dem Testen im Allgemeinen gewidmet. Danach erfolgt eine Einführung in die \textit{\nameref{sec:tdd}} im Kapitel \ref{sec:tdd}. \\
Da ein weiterer Schwerpunkt dieser Arbeit das Webframework Ruby on Rails\index{Ruby-on-Rails} ist, enthält das Kapitel \fullref{sec:ruby} einen kurzen Überblick über Ruby und Abschnitt \ref{sec:rails} einen Überblick über das darauf basierende Framework Rails. Ebenfalls werden einige Möglichkeiten für Tests in diesem Umfeld vorgestellt.\\
Das Kapitel \fullref{sec:metriken} zeigt auf, welche Möglichkeiten existieren, um den Erfolg der Testgetriebene\index{TDD}n Entwicklung praktisch anhand von Kennziffern nachzuweisen und den Prozess zu überwachen.

Im Kapitel \fullref{sec:auswahl} wird ausgeführt, wie die vorgestellten Werkzeuge und Methoden konkret in der pludoni GmbH eingesetzt werden sollten und welche zusätzlichen Faktoren es zu beachten gilt.
Im praktischen Kapitel \fullref{sec:awtdd} wird die Testgetriebene\index{TDD} Entwicklung mit Ruby on Rails\index{Ruby-on-Rails} an konkreten Beispielen erklärt und einige Sonderfälle, wie das Testen externer Abhängigkeiten im Unterabschnitt \ref{sec:awmock}, erläutert.
Danach schauen wir uns konkrete Ergebnisse im Kapitel \fullref{sec:auswertung} der Code-Metrik\index{Code-Metrik}en an und vergleichen diese mit anderen Projekten der pludoni GmbH und als Ausblick auch mit einigen bekannteren Rails\index{Ruby-on-Rails}-Projekten.

Zum Ende, im Abschnitt \fullref{sec:fazit}, werden Ergebnisse dieser Arbeit zusammengefasst und Vorschläge für eine weitere Forschung gegeben.

Im Appendix sind Hinweise für die Erstellung guter Tests aus der Literatur und die Anleitung, wie eine effektive Browsersimulation\index{Browser!Simulierter}sumgebung eingerichtet wird, hinterlegt. Am Ende der Arbeit erfolgt eine Auflistung der verwendeten Abbildungen, Quellcodes,  das Literaturverzeichnis sowie das Stichwortverzeichnis.

\paragraph{Hinweis zur Notation} In dieser Diplomarbeit werden einige Begriffe verwendet, die innerhalb der Arbeit nicht weiter erläutert werden, stattdessen aber im Glossar enthalten sind. Solche Einträge sind mit einem $^\uparrow$ markiert, z.B. \glossar{test}.\\
In einigen Fällen werden konkrete Techniken oder Methoden verwendet. Diese sind \textbf{fettgedruckt} dargestellt und werden bei erstmaliger Verwendung mit einer Fußnote und Quelle gekennzeichnet.\\
Zitiermarken im Text und die dazugehörige Ordnungsmarke im Literaturverzeichnis bestehen aus abgekürzten Verfasserbuchstaben plus Erscheinungsjahr in eckigen Klammern, z.B. \citep{beck_test_2002}.
