\newglossaryentry{test} {
  name={Test},
  description= {\index{Test}oder Testfall ist eine, meist automatisierte, Prüfung des Programmverhaltens bei definierten Eingabeparametern},
  plural=Tests
}

\newglossaryentry{testrunner} {
  name={Test-Runner},
  description= {oder Testtreiber) ist eine Software, um \glossarpl{test} aufzurufen und deren Ausführung zu überwachen und zu steuern}
}
\newglossaryentry{rake} {
  name={Rake},
  description= {ist ein Build-Programm, das im Gegensatz zu make mit Ruby programmiert wird. Es ist modular mit eigenen Tasks erweiterbar. Innerhalb von Rails dient es auch als weitere Schnittstelle zur Anwendung, um Wartungsaufgaben auszuführen (z.B. Datenbankbackup oder Cronjobs)}
}
\newglossaryentry{racktest} {
  name={RackTest},
  description= { ist eine Test-API für Rack-Anwendungen. Rack ist ein minimales Interface zur Kommunikation mit Webservern, Middlewares und Webframeworks, wie z.B. Rails. RackTest kommuniziert direkt mit dem Rack-Interface, verzichtet auf HTTP Traffic und ist damit sehr schnell.}
}

\newglossaryentry{testsuite} {
  name={Test-Suite},
  description= {ist eine Gruppe von mehreren Testfällen für eine Komponente oder das für gesamte System}
}

\newglossaryentry{fehler} {
  name={Software-Fehler},
  description= {oder Defekt, ist ein unerwartetes Verhalten der Software, der zu einem Versagen der Software führen kann},
text=Fehler,
first={Software-Fehler}
}
\newglossaryentry{quality} {
  name={Code-Qualität},
  description= {beinhaltet die Software-Qualitäten Lesbarkeit, Testbarkeit, Wartbarkeit, Erweiterbarkeit, geringe Komplexität},
}

\newglossaryentry{metrik} {
  name={Code-Metrik},
  plural={Code-Metriken},
  description={Eine Softwaremetrik ist das Ergebnis einer statischen oder dynamischen Codeanalyse zur Generierung von Informationen über den Quelltext. Beispiele: Testabdeckung, Anzahl Codezeilen, Anzahl \glossarpl{smell} pro Codezeile}
}
\newglossaryentry{refaktorisieren} {
  name = {Refaktorisieren},
  description = {Ist eine Modifikation des Programmcodes ohne Modifikation des externen Verhaltens um nicht-funktionale Eigenschaften des Quellcodes zu verbessern, wie z.B. Lesbarkeit, Wiederverwendbarkeit, Wartbarkeit}
}
\newglossaryentry{Testabdeckung} {
  name=Testabdeckung,
  description={auch: \index{Test!Testabdeckung}Testausführungsabdeckung, Überdeckungsgrad, Testfallabdeckung. Eine dynamische Code-Metrik\index{Code-Metrik}, die angibt, welche Codezeilen durch keinen Test abgedeckt wurde. Es wird unterschieden in die Stufen C0, C1 und C2 mit steigender Komplexität der Messung. Details im Abschnitt \ref{sec:metrics}},
  first = {Testabdeckung}
}
\newglossaryentry{TDD}{
  name={Testgetriebene Entwicklung},
  description={englisch Test Driven Development/Test Driven Design, Ausführlich dargelegt in Abschnitt \ref{sec:tdd}},
  text = TDD,
  first = {Testgetriebene Entwicklung}
}
\newglossaryentry{BDD}{
  name={ Behavior Driven Development},
  description={ (Verhaltensgetriebene Entwicklung). Umformulierung von TDD\index{TDD} zur Ausrichtung auf Businessprozesse. Das Vokabular zielt auf die Spezifikation von Erwartungen im Systemverhalten, anstatt Definition nachträglicher Tests}
}
\newglossaryentry{CRUD} {
  name={CRUD},
  description={Create, Read, Update, Delete sind die vier Basisoperationen, die auf persistente Speicherung ausgeführt werden können.},
  first={Create Read Update Delete (CRUD)}
}
\newglossaryentry{smell}{
  name={Code Smell},
  description={oder Bad Smell\index{Code-Smell}. Ist ein Anzeichen für eine suboptimale Quelltextstelle, die auch ein Hinweis auf ein größeres Designproblem sein kann. Oft auch ein Kandidat für ein Refaktoring. Informationen zu Smells und deren Refaktorisierung\index{Refaktorisierung} sind im Buch von M.Fowler zu finden \citep{fowler_refactoring_1999}},
  plural={Code Smells}
}
\newglossaryentry{ORM}{
  name={ORM},
  description={Objektrelationales Mapping, ist eine Persistenztechnik, um Objekte transparent in einer Datenbank\index{Datenbank} zu speichern und umgekehrt, Tabellenzeilen in Objekte wieder rückzuübersetzen},
  plural={ORMs},
  first={Objekt-relationales Mapping (ORM)}
}
\newglossaryentry{MVC}{
  name={MVC},
  description={Model-View-Control, ist ein \glossar{patterns}, das insbesondere bei GUI- und Web-Anwendungen beliebt ist. Rails\index{Ruby-on-Rails} basiert auf dem MVC-Muster},
  plural={MVCs},
  first={Model-View-Controller (MVC)}
}

\newglossaryentry{rails}{
  name={Ruby on Rails},
  description={ist ein auf der Programmiersprache Ruby basierendes Web-Framework und ist Gegenstand des Kapitel \ref{sec:rails}}
  text = Rails,
  first = Ruby on Rails
}

\newglossaryentry{Test-Double} {
  name={Test-Double},
  plural={Test-Doubles},
  description = {haben die Aufgabe, komplexe Objekte in einem isolierten Test \index{Test}zu simulieren, in dem statt komplexer Berechnungen oder externer Zugriffe konstante Werte geliefert werden. Vertreter dieser Test Doubles sind die Mock\index{Test-Double!Mock}s und Stub\index{Test-Double!Stub}s, siehe Abschnitt
 \ref{sec:mocks}}
}
\newglossaryentry{patterns}{
  name={Entwurfsmuster},
  description={ oder Design Patterns sind bewährte Vorlagen, um häufig wiederkehrende Probleme zu lösen. Weitere Informationen im Buch Design Patterns: Elements of Reusable Object-Oriented Software von Gamma, Helm, Johnson, Vlissides (Gang of Four)}
}
\newglossaryentry{RSS}{
  name={RSS},
  description={  RDF Site Summary, ist ein standardisierter XML-Dialekt zur maschinenlesbaren Verteilung und Veröffentlichung von Inhalten. Es existiert in den Versionen 0.9 bis 2.0.1, die sich nur in Details, wie z.B. Einbindung von Rich-Media (Podcasts, ..) und Namespaces unterscheiden}
}
\newglossaryentry{Test-Umgebung} {
  name={Test-Umgebung},
  description={ist eine spezielle Konfiguration der Hard- und Software, um Tests unter kontrollierten und bekannten Bedingungen auszuführen. Dies beinhaltet neben dem zu testenden Objekten (Object under Tes) auch sogenannte \glossarpl{Test-Double}\index{Test-Double} und den \glossar{testrunner}}
}
\newglossaryentry{gem} {
  name={Gem},
  plural={Gems},
  description={als Gem\index{Gem}s (zu deutsch: Edelsteine) werden in der Ruby-Gemeinschaft Bibliotheken Dritter bezeichnet, die in einem zentralem Repository lagern. Das Bekannteste dieser Repositorys ist rubygems.org. Nahezu alle Ruby-Bibliotheken lassen sich hier finden und innerhalb von Sekunden mittels des Kommandozeilenwerkzeuges gem installieren. Beispielsweise ließe sich \glossar{rails} mittels \texttt{gem install rails} installieren}
}

\newglossaryentry{DSL} {
  name={DSL},
  plural={DSLs},
  description={Eine Domain-spezifische Sprache ist eine auf eine spezielle Problemdomäne ausgerichtete Programmier- oder Spezifikationssprache. Vertreter sind z.B. SQL, die Statistiksprache R oder die Hardwarebeschreibungssprache VHDL. Sie stehen damit den Allzweck-Programmiersprachen gegenüber},
  first={Domain-Specific-Language (DSL)}
}


\newglossaryentry{CMS} {
  name={CMS},
  plural={CMSe},
  description={Content Management System (Inhaltsverwaltungssystem) ist eine, meist webbasierte Software, die es Nutzern ermöglicht einfach statische Inhalte anzulegen und zu bearbeiten},
first = {Content Management System}
}
\newcommand{\glossar}[1]{$^\uparrow$\gls{#1}}

\newcommand{\glossarpl}[1]{$^\uparrow$\glspl{#1}}
\glsaddall

\renewcommand{\glossarypreamble}{Im Folgenden werden einige oft verwendete Begriffe näher erläutert. Innerhalb des Hauptteils dieser Arbeit sind diese Begriffe mit einem $^\uparrow$ gekennzeichnet.}
% \renewcommand{\glossarymark}{Glossar}

\printglossary[toctitle=Glossar]

