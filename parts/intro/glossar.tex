\newglossaryentry{test} {
  name={Test},
  description= {oder Testfall ist eine, meist automatisierte, Prüfung des Programmverhaltens bei definierten Eingabeparametern},
  plural=Tests
}

\newglossaryentry{testrunner} {
  name={Test-Runner},
  description= {oder Testtreiber (test runner) ist eine Software, um \glossarpl{test} aufzurufen und deren Ausführung zu überwachen und steuern}
}%TODO Fritzsche Check

\newglossaryentry{testsuite} {
  name={Test-Suite},
  description= {ist eine Sammlung von mehreren Testfällen für eine Komponente oder das gesamte System},
parent=test
}%TODO Fritzsche Check

\newglossaryentry{fehler} {
  name={Software-Fehler},
  description= {oder Defekt, ist ein unerwartetes Verhalten der Software, der zu einem Versagen der Software führen kann},
text=Fehler,
first={Software-Fehler}
}
\newglossaryentry{quality} {
  name={Code-Qualität},
  description= {beinhaltet die Software-Qualitäten Lesbarkeit, Testbarkeit, Wartbarkeit, Erweiterbarkeit, Geringe Komplexität},
}% TODO Vorlesung Fritzsche?

\newglossaryentry{metriken} {
  name={Code-Metriken},
  description={Eine Softwaremetrik ist das Ergebnis einer statischen oder dynamischen Codeanalyse zur Generierung von Informationen über den Source-Code. Beispiele: Testabdeckung, Anzahl Codezeilen, Anzahl Bad Smells pro Codezeile.}
}
\newglossaryentry{refaktorisieren} {
  name = {Refaktorisieren},
  description = {Ist eine Modifikation des Programmcodes ohne Modifikation es externen Verhaltens um nicht-funktionale Eigenschaften des Quellcodes zu verbessern, wie z.B. Lesbarkeit, Wiederverwendbarkeit, Wartbarkeit}  
}
\newglossaryentry{Testabdeckung} {
  name=Testabdeckung,
  description={auch: Testausführungsabdeckung, Überdeckungsgrad, Testfallabdeckung. Eine dynamische Code-Metrik, die angibt, welche Codezeilen durch keinen Test abgedeckt wurde. Es wird unterschieden in die Stufen C0, C1 und C2 mit steigender Komplexität der Messung. Details im Abschnitt \ref{sec:metrics}},
  first = {Testausführungsabdeckung}
}
\newglossaryentry{TDD}{
  name={Testgetriebene Entwicklung},
  description={englisch Test-Driven-Development/Test-Driven-Design, Ausführlich dargelegt in Abschnitt \ref{sec:tdd}},
  text = TDD,
  first = {Testgetriebene Entwicklung}
}
\newglossaryentry{BDD}{
  name={ Behavior-Driven-Development},
  description={ (Verhaltensgetriebene Entwicklung). Umformulierung von TDD zur Ausrichtung auf Businessprozesse. Das Vokabular zielt auf die Spezifikation von Erwartungen im Systemverhalten, anstatt Definition nachträglicher Tests.}
}
\newglossaryentry{smell}{
  name={Bad Smell},
  description={oder Code-Smell. Ist ein Anzeichen für eine suboptimale Code-Stelle, die auch ein Hinweis auf ein größeres Design-Problem sein kann. Oft auch ein Kandidat für ein Refaktoring},
  plural={Bad Smells}
%TODO Verweis auf Fowler
}

\newglossaryentry{rails}{
  name={Ruby on Rails},
  description={ist ein auf der Programmiersprache Ruby basierendes Web-Framework und ist Gegenstand des Kapitel \ref{sec:rails}}
  text = Rails,
  first = Ruby on Rails
}

\newglossaryentry{Test Double} {
  name={Test Double},
  plural={Test Doubles},
  description = {haben die Aufgabe, komplexe Objekte in einem isolierten Test zu simulieren, in dem statt komplexer Berechnungen oder externer Zugriffe konstante Werte geliefert werden. Vertreter dieser Test Doubles sind die Mocks und Stubs, siehe Abschnitt
 \ref{sec:mocks}}
}
\newglossaryentry{patterns}{
  name={Entwurfsmuster},
  description={ oder Design Patterns sind bewährte Vorlagen, um häufig wiederkehrende Probleme zu lösen. Weitere Informationen im Buch Design Patterns: Elements of Reusable Object-Oriented Software von Gamme, Helm, Johnson, Vlissides (Gang of Four).}
}
\newglossaryentry{RSS}{
  name={RSS},
  description={  RDF Site Summary, ist ein standardisierter XML-Dialekt zur maschinenlesbaren Verteilung und Veröffentlichung von Inhalten. Es existiert in den Versionen 0.9 bis 2.0.1, die sich nur in Details, wie z.B. Einbindung von Rich-Media (Podcasts, ..) und Namespaces unterscheiden}
}
\newglossaryentry{Test-Umgebung} {
  name={Test-Umgebung},
  description={ist eine spezielle Konfiguration der Hard- und Software, um Tests unter kontrollierten und bekannten Bedingungen auszuführen. Dies beinhaltet neben dem zu testenden Objekten (Obeject under Tes) auch sogenannte \glossarpl{Test Double} und den \glossar{testrunner}.}
}
\newcommand{\glossar}[1]{$^\uparrow$\gls{#1}}

\newcommand{\glossarpl}[1]{$^\uparrow$\glspl{#1}}
\glsaddall

\renewcommand{\glossarypreamble}{Im Folgenden werden einige oft-verwendete Begriffe näher erläutert. Innerhalb des Hauptteils dieser Arbeit sind diese Begriffe mit einem $^\uparrow$ gekennzeichnet}
% \renewcommand{\glossarymark}{Glossar}

\printglossary[toctitle=Glossar]

