\section{Implementierung von Controller-Tests (functional tests)}

Neben den Unittests stellt Ruby on Rails eine weitere Testart nativ bereit. Technisch gesehen handelt es sich bei diesen Functional Tests aber auch um Unittests, da deren Testobjekt ein Controller ist. 
Ein Controller hat bei Ruby on Rails die Aufgabe, Anfragen für bestimmte Routen, also Web-Adressen, anzunehmen, die Arbeit an eine Modelklasse auszulagern, und eine View aufzurufen, die letztendlich HTML-Code generiert.

Im ersten Beispiel wollen wir testen, dass ein Gast-Nutzer, also z.B. ein Bewerber, eine sichtbare Stellenanzeige aufrufen darf (visible = true). Hierbei verwenden wir wieder unser oben definiertes Fixture für einen gültigen Job.
\begin{lstlisting}
/test/functional/jobs_controller_test.rb 
require 'test_helper'

class JobsControllerTest < ActionController::TestCase
  test "Gast Nutzer kann Stellen betrachten" do
    session[:user_id] = nil
    job = jobs(:valid_job)
    
    get :show, :id => job.id
    
    assert_response :success
    assert_equal job, assigns(:job)
  end
end
\end{lstlisting}
\tddred
Zuerst loggen wir jeglichen Nutzer aus, der eventuell eingeloggt war, dann laden wir das Fixture und führen einen simulierten HTTP Request auf die Detailansicht der Stellenanzeige aus (Die Aktion "`show"' mit der ID des Jobs).
Nun erwarten wir, dass wir einen HTTP-Status Code 200 (success) erhalten, und dass der Controller eine Variable "`@jobs"' bereitstellt, die mit unserem Fixture identisch ist.

Die Implementation dieser Anforderung könnte wie folgt umgesetzt werden:
\begin{lstlisting}
# app/controllers/jobs_controller.rb
class JobsController < ApplicationController
  ...
  def show
    @job = Job.first
  end
  ...
  
end
\end{lstlisting}
\tddgreen
Das Laden des erstem Jobs aus unserer Datenbank genügt zum Erfüllen der Anforderungen, und ist ein schneller Weg, den Test bestehen zu lassen. Allerdings handelt es sich hierbei um eine Fake-Implementierung, da zwar unser Test erfüllt wird, aber die Anwendung nicht das macht, was man sich erhofft hat. Solche Zwischenschritte sind aber ausdrücklich vorgesehen, da das Ziel ist, so schnell wir möglich einen funktionierenden Test zu erhalten mit dem man arbeiten kann.

Wenn wir nun weitere Tests schreiben, so wird es immer schwieriger, die Fake-Implementierung beizubehalten, und früher oder später wird eine korrekte Implementierung folgen. 
Aber wir können auch die nun folgende Refaktorisierungsphase nutzen, um diesen Makel zu beseitigen:
\tddrefactor
\begin{lstlisting}
# app/controllers/jobs_controller.rb
def show
  @job = Job.find(params[:id])
end
\end{lstlisting}

Nun wollen wir testen, ob ein Gast von einer nicht-sichtbaren Stellenanzeige weitergeleitet wird und einen Hinweis erhält.

\begin{lstlisting}
test "Gast Nutzer kann nicht-sichtbare Stellen nicht betrachten" do
  session[:user_id] = nil
  job = jobs(:invisible_job)
  
  get :show, :id => job.id

  assert_response :redirect
  assert flash[:notice].present?
end
\end{lstlisting}
\tddred

Wir laden unser zweites definiertes Fixture, dass eine unsichtbaren Stellenanzeige. Dieses mal erwarten wir einen HTTP Statuscode 301 (Redirect), und dass unser Controller eine Hinweisnachricht generiert.

\begin{lstlisting}
def show
  @job = Job.find(params[:id])
  if not @job.visible?
    redirect_to root_path, :notice => "Diese Stelle ist zur Zeit nicht sichtbar"
  end
end
\end{lstlisting}
\tddgreen
Falls der aktuelle Job nicht sichtbar ist, dann erfolgt eine Weiterleitung auf die Startseite und die Bereitstellung des Hinweistextes.
\tddrefactor
Da auch hier der Quelltext wieder sehr kurz ist, ist ein Refaktorisieren nicht notwendig.

Nun möchten wir, dass ein Kunde dieser Anwendung, also ein Unternehmen seine Stellenanzeige betrachten kann, auch wenn diese unsichtbar ist, sei es aus Gründen der Archivierung als auch der Vorbereitung für eine Veröffentlichung.

\begin{lstlisting}
test "Ein Kunde darf aber seine unsichtbaren Jobs betrachten" do
  job = jobs(:invisible_job)
  session[:user_id] = job.user_id
  
  get :show, :id => job.id

  assert_response :success
end
\end{lstlisting}
\tddred
Über die globale Session Variable simulieren wir das Einloggen durch setzen der UserID in dieses Array. Die genaue Implementation hängt natürlich davon ab, wie man die Authentifizierung implementiert hat, oder welche Bibliothek man verwendet. In diesem Beispiel sei darauf hingewiesen, dass die Definition, ob ein Nutzer eingeloggt ist oder nicht, davon abhängt, ob in seiner Session-Variable eine UserID enthalten ist.

\begin{lstlisting}
def show
  @job = Job.find(params[:id])
  if !@job.visible? and @job.user != User.find(session[:user_id])
    redirect_to root_path, :notice => "Diese Stelle ist zur Zeit nicht sichtbar"
  end
end
\end{lstlisting}
\tddgreen
Wir lösen diesen Test damit, dass wir in der Weiterleitungsbedingung prüfen, ob der betrachtende Nutzer und der Eigentümer des Jobs gleich sind.

Nun können wir refaktorisieren. Was auffällt, ist z.B. dass unser Controller und die Klasse Job nicht lose gekoppelt sind, da die Bedingung zweimal auf Attribute des Jobs zurückgreift. Eine Lösung wäre die Auslagerung in die Modelklasse von Job:
\tddrefactor
\begin{lstlisting}
# app/models/job.rb
class Job < ActiveRecord::Base
  ...
  def visible_for_user?(user)
    self.visible and self.user != user
  end 
end

# app/controllers/jobs_controller.rb 
def show
  @job = Job.find(params[:id])
  unless @job.visible_for_user?(User.find(session[:user_id]))
    redirect_to root_path, :notice => "Diese Stelle ist zur Zeit nicht sichtbar"
  end
end
\end{lstlisting}
Ebenfalls wurde das syntaktische Element "`unless"' verwendet, welches ein Alias für "`if not"' ist.
Weiterhin könnte die Suche nach dem aktuell eingeloggten Nutzer in eine für alle Controller sichtbare Funktion ausgegliedert werden
\tddrefactor
\begin{lstlisting}

# app/controllers/application_controller.rb
...
  def current_user
    User.find(session[:user_id]
  end

# app/controllers/jobs_controller.rb 
def show
  @job = Job.find(params[:id])
  unless @job.visible_for_user? current_user
    redirect_to root_path, :notice => "Diese Stelle ist zur Zeit nicht sichtbar"
  end
end
\end{lstlisting}


% Funktionale Tests sind in Rails häufig recht einfach gehalten, da das Ziel vergo
\borderquote{Skinny Controller, Fat Model [...] Try to keep your controller actions and views as slim as possible.}{Jamis Buck, Programmierer bei 37signals}
Funktionale Tests und deren Controllerimplementierungen sind häufig nischt länger als ein paar Zeilen. Qua Konventation des \glossar{MVC}-Patterns und Rails sollen komplexe Abläufe in den Modellen oder auch in Bibliotheken stattfinden. 
Die Aspekte, die üblicherweise bei Controllern getestet werden, sind:
\begin{itemize}
 \item HTTP Statuscodes und Weiterleitungen,
 \item das Vorhandensein von Statusmeldungen, gennant "`Flash"' Messages
 \item dass ein bestimmtes Template geladen wird
 \item dass Instanzvariablen gesetzt werden, die die View später darstellen sollen
 \item falls man Viewtests mit einschließt, dann wird u.U. auch auf das Vorhandensein von bestimmten HTML-Elementen in der am Ende generierten View getestet. Z.B. möchte man wissen, ob das Überschriftenelement "`h1"' dem Job-Titel entspricht, wenn die Detailansicht eines Jobs aufgerufen wird.
\end{itemize}