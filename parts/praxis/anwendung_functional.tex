\section{Implementierung von Controller\index{Ruby-on-Rails!Controller}-Tests (functional tests)}
\epigraph{Skinny Controller\index{Ruby-on-Rails!Controller}, Fat Model [...] Try to keep your controller actions and views as slim as possible.}{Jamis Buck, Programmierer bei 37signals}


Neben den Unittest\index{Test!Unittest}s stellt Ruby on Rails\index{Ruby-on-Rails} eine weitere Testart nativ bereit. Technisch gesehen handelt es sich bei diesen Functional Tests aber auch um Unittests, da deren Testobjekt eine Klasse, der Controller\index{Ruby-on-Rails!Controller}, ist. 
Ein Controller\index{Ruby-on-Rails!Controller} hat bei Ruby on Rails\index{Ruby-on-Rails} die Aufgabe, Anfragen für bestimmte Routen, also Web-Adressen, anzunehmen, die Arbeit an eine Modelklasse auszulagern und eine View\index{Ruby-on-Rails!View} aufzurufen, die letztendlich HTML-Code generiert.

Im ersten Beispiel wollen wir testen, dass ein Gast-Nutzer, also z.B. ein Bewerber, eine sichtbare Stellenanzeige aufrufen darf (\texttt{visible = true}). Hierbei verwenden wir wieder unser oben definiertes Fixture für einen gültigen Job.
% SNIPPET: 
%                                                                                                                                                                    
% /test/functional/jobs_controller_test.rb                                                                                                                           
% require 'test_helper'                                                                                                                                              
%                                                                                                                                                                    
% class JobsControllerTest < ActionController::TestCase                                                                                                              
%   test "Gast Nutzer kann Stellen betrachten" do                                                                                                                    
%     session[:user_id] = nil                                                                                                                                        
%     job = jobs(:valid_job)                                                                                                                                         
%                                                                                                                                                                    
%     get :show, :id => job.id                                                                                                                                       
%                                                                                                                                                                    
%     assert_response :success                                                                                                                                       
%     assert_equal job, assigns(:job)                                                                                                                                
%   end                                                                                                                                                              
% end                                                                                                                                                                
\begin{ruby}[label=test/functional/jobs\_controller\_test.rb]
\PY{n+nb}{require} \PY{l+s+s1}{'test\PYZus{}helper'}

\PY{k}{class} \PY{n+nc}{JobsControllerTest} \PY{o}{<} \PY{n+no}{ActionController}\PY{o}{::}\PY{n+no}{TestCase}
  \PY{n+nb}{test} \PY{l+s+s2}{"}\PY{l+s+s2}{Gast Nutzer kann Stellen betrachten}\PY{l+s+s2}{"} \PY{k}{do}
    \PY{n}{session}\PY{o}{[}\PY{l+s+ss}{:user\PYZus{}id}\PY{o}{]} \PY{o}{=} \PY{k+kp}{nil}
    \PY{n}{job} \PY{o}{=} \PY{n}{jobs}\PY{p}{(}\PY{l+s+ss}{:valid\PYZus{}job}\PY{p}{)}
    
    \PY{n}{get} \PY{l+s+ss}{:show}\PY{p}{,} \PY{l+s+ss}{:id} \PY{o}{=}\PY{o}{>} \PY{n}{job}\PY{o}{.}\PY{n}{id}
    
    \PY{n}{assert\PYZus{}response} \PY{l+s+ss}{:success}
    \PY{n}{assert\PYZus{}equal} \PY{n}{job}\PY{p}{,} \PY{n}{assigns}\PY{p}{(}\PY{l+s+ss}{:job}\PY{p}{)}
  \PY{k}{end}
\PY{k}{end}
\end{ruby}
\codecaption{JobsController-Test: Gast Nutzer darf Stellen betrachten}

\tddred
Zuerst loggen wir jeglichen Nutzer aus, der eventuell eingeloggt war, dann laden wir das Fixture und führen einen simulierten HTTP Request auf die Detailansicht der Stellenanzeige aus (Die Aktion \texttt{show} mit der ID des Jobs).
Nun erwarten wir, dass wir einen HTTP-Status Code 200 (Erfolg) erhalten und dass der Controller\index{Ruby-on-Rails!Controller} eine Variable \texttt{@jobs} bereitstellt, die mit unserem Fixture identisch ist.

Die Implementation dieser Anforderung könnte wie folgt umgesetzt werden:
% -------------------------------
% SNIPPET: 
%                                                                                                                                                                    
% # app/controllers/jobs_controller.rb                                                                                                                               
% class JobsController < ApplicationController                                                                                                                       
%   ...                                                                                                                                                              
%   def show                                                                                                                                                         
%     @job = Job.first                                                                                                                                               
%   end                                                                                                                                                              
%   ...                                                                                                                                                              
%                                                                                                                                                                    
% end                                                                                                                                                                
\begin{ruby}[label=app/controllers/jobs\_controller.rb]
\PY{k}{class} \PY{n+nc}{JobsController} \PY{o}{<} \PY{n+no}{ApplicationController}
  \PY{o}{.}\PY{n}{.}\PY{o}{.}
  \PY{k}{def} \PY{n+nf}{show}
    \PY{n+nv+vi}{@job} \PY{o}{=} \PY{n+no}{Job}\PY{o}{.}\PY{n}{first}
  \PY{k}{end}
  \PY{o}{.}\PY{n}{.}\PY{o}{.}
\PY{k}{end}
\end{ruby}
\codecaption{JobsController: Erfüllung des Tests durch eine Fake-Implementierung}

\tddgreen
Das Laden des erstem Jobs aus unserer Datenbank genügt zum Erfüllen der Anforderungen und ist ein schneller Weg, den Test \index{Test}bestehen zu lassen. Allerdings handelt es sich hierbei um eine \textbf{Fake-Implementierung}, da zwar unser Test erfüllt wird, aber die Anwendung nicht das macht, was man sich erhofft hat. Solche Zwischenschritte sind aber ausdrücklich vorgesehen, da das Ziel ist, so schnell wir möglich einen funktionierenden Test zu erhalten mit dem man arbeiten kann.

Wenn wir nun weitere Tests schreiben, so wird es immer schwieriger, die Fake-Implementierung beizubehalten und früher oder später wird eine korrekte Implementierung folgen. 
Aber wir können auch die nun folgende Refaktorisierung\index{Refaktorisierung}sphase nutzen, um diesen Makel zu beseitigen:
\tddrefactor
% -------------------------------
% SNIPPET: 
%                                                                                                                                                                    
% # app/controllers/jobs_controller.rb                                                                                                                               
% def show                                                                                                                                                           
%   @job = Job.find(params[:id])                                                                                                                                     
% end                                                                                                                                                                
\begin{ruby}[label=app/controllers/jobs\_controller.rb]
\PY{k}{def} \PY{n+nf}{show}
  \PY{n+nv+vi}{@job} \PY{o}{=} \PY{n+no}{Job}\PY{o}{.}\PY{n}{find}\PY{p}{(}\PY{n}{params}\PY{o}{[}\PY{l+s+ss}{:id}\PY{o}{]}\PY{p}{)}
\PY{k}{end}
\end{ruby}
\codecaption{JobsController: Ersetzung der Fake-Implementierung}


Nun wollen wir testen, ob ein Gast von einer nicht-sichtbaren Stellenanzeige weitergeleitet wird und einen Hinweis erhält.

% -------------------------------
% SNIPPET: 
%                                                                                                                                                                    
% test "Gast Nutzer kann nicht-sichtbare Stellen nicht betrachten" do                                                                                                
%   session[:user_id] = nil                                                                                                                                          
%   job = jobs(:invisible_job)                                                                                                                                       
%                                                                                                                                                                    
%   get :show, :id => job.id                                                                                                                                         
%                                                                                                                                                                    
%   assert_response :redirect                                                                                                                                        
%   assert flash[:notice].present?                                                                                                                                   
% end                                                                                                                                                                
\begin{ruby}[label=test/functional/jobs\_controller\_test.rb]
\PY{n+nb}{test} \PY{l+s+s2}{"}\PY{l+s+s2}{Gast Nutzer kann nicht-sichtbare Stellen nicht betrachten}\PY{l+s+s2}{"} \PY{k}{do}
  \PY{n}{session}\PY{o}{[}\PY{l+s+ss}{:user\PYZus{}id}\PY{o}{]} \PY{o}{=} \PY{k+kp}{nil}
  \PY{n}{job} \PY{o}{=} \PY{n}{jobs}\PY{p}{(}\PY{l+s+ss}{:invisible\PYZus{}job}\PY{p}{)}
  
  \PY{n}{get} \PY{l+s+ss}{:show}\PY{p}{,} \PY{l+s+ss}{:id} \PY{o}{=}\PY{o}{>} \PY{n}{job}\PY{o}{.}\PY{n}{id}

  \PY{n}{assert\PYZus{}response} \PY{l+s+ss}{:redirect}
  \PY{n}{assert} \PY{n}{flash}\PY{o}{[}\PY{l+s+ss}{:notice}\PY{o}{]}\PY{o}{.}\PY{n}{present?}
\PY{k}{end}
\end{ruby}
\codecaption{JobsController-Test: Gast Nutzer dürfen nicht-sichtbare Stellen nicht betrachten}

\tddred

Wir laden unser zweites definiertes Fixture, dass eine unsichtbaren Stellenanzeige. Dieses mal erwarten wir einen HTTP Statuscode 301 (Redirect/Weiterleitung) und dass unser Controller\index{Ruby-on-Rails!Controller} eine Hinweisnachricht generiert.

% -------------------------------
% SNIPPET: 
%                                                                                                                                                                    
% def show                                                                                                                                                           
%   @job = Job.find(params[:id])                                                                                                                                     
%   if not @job.visible?                                                                                                                                             
%     redirect_to root_path, :notice => "Diese Stelle ist zur Zeit nicht sichtbar"                                                                                   
%   end                                                                                                                                                              
% end                                                                                                                                                                
\begin{ruby}[label=app/controllers/jobs\_controller.rb]
\PY{k}{def} \PY{n+nf}{show}
  \PY{n+nv+vi}{@job} \PY{o}{=} \PY{n+no}{Job}\PY{o}{.}\PY{n}{find}\PY{p}{(}\PY{n}{params}\PY{o}{[}\PY{l+s+ss}{:id}\PY{o}{]}\PY{p}{)}
  \PY{k}{if} \PY{o+ow}{not} \PY{n+nv+vi}{@job}\PY{o}{.}\PY{n}{visible?}
    \PY{n}{redirect\PYZus{}to} \PY{n}{root\PYZus{}path}\PY{p}{,} \PY{l+s+ss}{:notice} \PY{o}{=}\PY{o}{>} \PY{l+s+s2}{"}\PY{l+s+s2}{Diese Stelle ist zur Zeit nicht sichtbar}\PY{l+s+s2}{"}
  \PY{k}{end}
\PY{k}{end}
\end{ruby}
\codecaption{JobsController: Weiterleitung, falls ein Job nicht sichtbar ist}

\tddgreen
Falls der aktuelle Job nicht sichtbar ist, dann erfolgt eine Weiterleitung auf die Startseite und die Bereitstellung des Hinweistextes.
\tddrefactor
Da auch hier der Quelltext wieder sehr kurz ist, ist ein Refaktorisieren nicht notwendig.

Nun möchten wir, dass ein Kunde dieser Anwendung, also ein Unternehmen seine Stellenanzeige betrachten kann, auch wenn diese unsichtbar ist, sei es aus Gründen der Archivierung als auch der Vorbereitung für eine Veröffentlichung.

% SNIPPET: 
%                                                                                                                                                                    
% test "Ein Kunde darf aber seine unsichtbaren Jobs betrachten" do                                                                                                   
%   job = jobs(:invisible_job)                                                                                                                                       
%   session[:user_id] = job.user_id                                                                                                                                  
%                                                                                                                                                                    
%   get :show, :id => job.id                                                                                                                                         
%                                                                                                                                                                    
%   assert_response :success                                                                                                                                         
% end                                                                                                                                                                
\begin{ruby}[label=test/functional/job\_controller\_test.rb]
\PY{n+nb}{test} \PY{l+s+s2}{"}\PY{l+s+s2}{Ein Kunde darf aber seine unsichtbaren Jobs betrachten}\PY{l+s+s2}{"} \PY{k}{do}
  \PY{n}{job} \PY{o}{=} \PY{n}{jobs}\PY{p}{(}\PY{l+s+ss}{:invisible\PYZus{}job}\PY{p}{)}
  \PY{n}{session}\PY{o}{[}\PY{l+s+ss}{:user\PYZus{}id}\PY{o}{]} \PY{o}{=} \PY{n}{job}\PY{o}{.}\PY{n}{user\PYZus{}id}
  
  \PY{n}{get} \PY{l+s+ss}{:show}\PY{p}{,} \PY{l+s+ss}{:id} \PY{o}{=}\PY{o}{>} \PY{n}{job}\PY{o}{.}\PY{n}{id}

  \PY{n}{assert\PYZus{}response} \PY{l+s+ss}{:success}
\PY{k}{end}
\end{ruby}
\codecaption{JobController-Test: Neuer Testfall}

\tddred
Über die globale Session Variable simulieren wir das Einloggen durch setzen der User-ID in dieses Array. Die genaue Implementation hängt natürlich davon ab, wie man die Authentifizierung implementiert hat oder welche Bibliothek man verwendet. In diesem Beispiel sei darauf hingewiesen, dass die Definition, ob ein Nutzer eingeloggt ist oder nicht, davon abhängt, ob in seiner Session-Variable eine User-ID enthalten ist.

% SNIPPET: 
%                                                                                                                                                                    
% def show                                                                                                                                                           
%   @job = Job.find(params[:id])                                                                                                                                     
%   if !@job.visible? and @job.user != User.find(session[:user_id])                                                                                                  
%     redirect_to root_path, :notice => "Diese Stelle ist zur Zeit nicht sichtbar"                                                                                   
%   end                                                                                                                                                              
% end                                                                                                                                                                
\begin{ruby}[label=app/controllers/job\_controller.rb]
\PY{k}{def} \PY{n+nf}{show}
  \PY{n+nv+vi}{@job} \PY{o}{=} \PY{n+no}{Job}\PY{o}{.}\PY{n}{find}\PY{p}{(}\PY{n}{params}\PY{o}{[}\PY{l+s+ss}{:id}\PY{o}{]}\PY{p}{)}
  \PY{k}{if} \PY{o}{!}\PY{n+nv+vi}{@job}\PY{o}{.}\PY{n}{visible?} \PY{o+ow}{and} \PY{n+nv+vi}{@job}\PY{o}{.}\PY{n}{user} \PY{o}{!=} \PY{n+no}{User}\PY{o}{.}\PY{n}{find}\PY{p}{(}\PY{n}{session}\PY{o}{[}\PY{l+s+ss}{:user\PYZus{}id}\PY{o}{]}\PY{p}{)}
    \PY{n}{redirect\PYZus{}to} \PY{n}{root\PYZus{}path}\PY{p}{,} \PY{l+s+ss}{:notice} \PY{o}{=}\PY{o}{>} \PY{l+s+s2}{"}\PY{l+s+s2}{Diese Stelle ist zur Zeit nicht sichtbar}\PY{l+s+s2}{"}
  \PY{k}{end}
\PY{k}{end}
\end{ruby}
\codecaption{JobsController: Implementierung der Weiterleitung, falls Stelle unsichtbar}

\tddgreen
Wir lösen diesen Test \index{Test}damit, dass wir in der Weiterleitungsbedingung prüfen, ob der betrachtende Nutzer und der Eigentümer des Jobs gleich sind.

Nun können wir refaktorisieren. Was auffällt, ist z.B. dass unser Controller\index{Ruby-on-Rails!Controller} und die Klasse Job nicht lose gekoppelt sind, da die Bedingung zweimal auf Attribute des Jobs zurückgreift. Eine Lösung wäre die Auslagerung in die Modelklasse von Job:
\tddrefactor
% SNIPPET: 
%                                                                                                                                                                    
% # app/models/job.rb                                                                                                                                                
% class Job < ActiveRecord::Base                                                                                                                                     
%   ...                                                                                                                                                              
%   def visible_for_user?(user)                                                                                                                                      
%     self.visible and self.user != user                                                                                                                             
%   end                                                                                                                                                              
% end                                                                                                                                                                
%                                                                                                                                                                    
% # app/controllers/jobs_controller.rb                                                                                                                               
% def show                                                                                                                                                           
%   @job = Job.find(params[:id])                                                                                                                                     
%   unless @job.visible_for_user?(User.find(session[:user_id]))                                                                                                      
%     redirect_to root_path, :notice => "Diese Stelle ist zur Zeit nicht sichtbar"                                                                                   
%   end                                                                                                                                                              
% end                                                                                                                                                                
\begin{ruby}[label=app/models/job.rb]
\PY{k}{class} \PY{n+nc}{Job} \PY{o}{<} \PY{n+no}{ActiveRecord}\PY{o}{::}\PY{n+no}{Base}
  \PY{o}{.}\PY{n}{.}\PY{o}{.}
  \PY{k}{def} \PY{n+nf}{visible\PYZus{}for\PYZus{}user?}\PY{p}{(}\PY{n}{user}\PY{p}{)}
    \PY{n+nb}{self}\PY{o}{.}\PY{n}{visible} \PY{o+ow}{and} \PY{n+nb}{self}\PY{o}{.}\PY{n}{user} \PY{o}{!=} \PY{n}{user}
  \PY{k}{end} 
\PY{k}{end}
\end{ruby}
\codecaption{Job: Einführung einer Methode zur Bestimmung der Sichtbarkeit eines Jobs im Job-Modell}
\begin{ruby}[label=app/controller/jobs\_controller.rb]
\PY{k}{def} \PY{n+nf}{show}
  \PY{n+nv+vi}{@job} \PY{o}{=} \PY{n+no}{Job}\PY{o}{.}\PY{n}{find}\PY{p}{(}\PY{n}{params}\PY{o}{[}\PY{l+s+ss}{:id}\PY{o}{]}\PY{p}{)}
  \PY{k}{unless} \PY{n+nv+vi}{@job}\PY{o}{.}\PY{n}{visible\PYZus{}for\PYZus{}user?}\PY{p}{(}\PY{n+no}{User}\PY{o}{.}\PY{n}{find}\PY{p}{(}\PY{n}{session}\PY{o}{[}\PY{l+s+ss}{:user\PYZus{}id}\PY{o}{]}\PY{p}{)}\PY{p}{)}
    \PY{n}{redirect\PYZus{}to} \PY{n}{root\PYZus{}path}\PY{p}{,} \PY{l+s+ss}{:notice} \PY{o}{=}\PY{o}{>} \PY{l+s+s2}{"}\PY{l+s+s2}{Diese Stelle ist zur Zeit nicht sichtbar}\PY{l+s+s2}{"}
  \PY{k}{end}
\PY{k}{end}
\end{ruby}
\codecaption{JobsController: Anwendung der Refaktoriserung}

Ebenfalls wurde das syntaktische Element \texttt{unless} verwendet, welches ein Alias für \texttt{if not} ist.
Weiterhin könnte die Suche nach dem aktuell eingeloggten Nutzer in eine für alle Controller\index{Ruby-on-Rails!Controller} sichtbare Funktion ausgegliedert werden
\tddrefactor

% SNIPPET: 
%                                                                                                                                                                    
%                                                                                                                                                                    
% # app/controllers/application_controller.rb                                                                                                                        
% ...                                                                                                                                                                
%   def current_user                                                                                                                                                 
%     User.find(session[:user_id]                                                                                                                                    
%   end                                                                                                                                                              
%                                                                                                                                                                    
% # app/controllers/jobs_controller.rb                                                                                                                               
% def show                                                                                                                                                           
%   @job = Job.find(params[:id])                                                                                                                                     
%   unless @job.visible_for_user? current_user                                                                                                                       
%     redirect_to root_path, :notice => "Diese Stelle ist zur Zeit nicht sichtbar"                                                                                   
%   end                                                                                                                                                              
% end                                                                                                                                                                
\begin{ruby}[label=app/controllers/application\_controller.rb]
\PY{o}{.}\PY{n}{.}\PY{o}{.}
  \PY{k}{def} \PY{n+nf}{current\PYZus{}user}
    \PY{n+no}{User}\PY{o}{.}\PY{n}{find}\PY{p}{(}\PY{n}{session}\PY{o}{[}\PY{l+s+ss}{:user\PYZus{}id}\PY{o}{]}
  \PY{k}{end}
\end{ruby}
\codecaption{ApplicationController: Einführung einer controllerglobalen Methode current\_user}


\begin{ruby}[label=app/controllers/jobs\_controller.rb ]

\PY{c+c1}{# app/controllers/jobs\PYZus{}controller.rb }
\PY{k}{def} \PY{n+nf}{show}
  \PY{n+nv+vi}{@job} \PY{o}{=} \PY{n+no}{Job}\PY{o}{.}\PY{n}{find}\PY{p}{(}\PY{n}{params}\PY{o}{[}\PY{l+s+ss}{:id}\PY{o}{]}\PY{p}{)}
  \PY{k}{unless} \PY{n+nv+vi}{@job}\PY{o}{.}\PY{n}{visible\PYZus{}for\PYZus{}user?} \PY{n}{current\PYZus{}user}
    \PY{n}{redirect\PYZus{}to} \PY{n}{root\PYZus{}path}\PY{p}{,} \PY{l+s+ss}{:notice} \PY{o}{=}\PY{o}{>} \PY{l+s+s2}{"}\PY{l+s+s2}{Diese Stelle ist zur Zeit nicht sichtbar}\PY{l+s+s2}{"}
  \PY{k}{end}
\PY{k}{end}
\end{ruby}
\codecaption{JobsController: Nutzung der neuen Methode current\_user}



\paragraph{Zusammenfassung}
Funktionale Tests und deren Controller\index{Ruby-on-Rails!Controller}-Implementierungen sind häufig nicht länger als ein paar Zeilen. Qua Konventation des \glossar{MVC}-Patterns und Rails\index{Ruby-on-Rails} sollen komplexe Abläufe in den Modell\index{Ruby-on-Rails!Modell}en oder auch in Bibliotheken stattfinden. 
Die Aspekte, die üblicherweise bei Controller\index{Ruby-on-Rails!Controller}n getestet werden, sind:
\begin{itemize}
 \item HTTP Statuscodes und Weiterleitungen,
 \item das Vorhandensein von Statusmeldungen, gennant "`Flash"'-Messages
 \item dass ein bestimmtes Template geladen wird
 \item dass Instanzvariablen gesetzt werden, die die View\index{Ruby-on-Rails!View} später darstellen sollen
 \item falls man View\index{Ruby-on-Rails!View}tests mit einschließt, dann wird u.U. auch auf das Vorhandensein von bestimmten HTML-Elementen in der am Ende generierten View getestet. Z.B. möchte man wissen, ob das Überschriftenelement \texttt{h1} dem Job-Titel entspricht, wenn die Detailansicht eines Jobs aufgerufen wird.
\end{itemize}

Intern nutzen Controller\index{Ruby-on-Rails!Controller}-Tests viele Stub\index{Test-Double!Stub}s, um HTTP-Anfragen und Antworten durch eigene TestKlassen zu ersetzen. Sie sind damit sehr schnell, testen aber nicht alle Aspekte einer HTTP-Anfrage an die Anwendung (z.B. Cookie oder Routing). Auch ist das Testen von mehreren Controllern, um z.B. einen Ablauf wie Einloggen $\to$ Bestellen nachzubilden, in einem Functional-Test nicht vorgesehen.