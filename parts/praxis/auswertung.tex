\section{Auswertung}

* Fortschritt in der Entwicklung

* Diskussion der Metriken


\subsection{Eigenschaften erfolgreicher Tests}

TODO Ueberarbeitung und merging mit MSDNA Tipps und eigene Erfahrungen
%TODO 

Das Vorhandensein von zahlreichen Tests reicht nicht, um das Testen erfolgreich abzuschließen. Zur Beurteilung der Brauchbarkeit einer Testsuite genügen die folgenden Kriterien \cite[S.272-279]{rappin_rails_2011}.

\begin{description}
 \item[Unabhängigkeit (Independence)] Ein Test ist unabhängig, falls er nicht durch andere Tests beeinflusst wird. Auch die Reihenfolge, in der die Tests ausgeführt werden, darf auf das Ergebnis keinen Einfluss üben. Siehe auch \citep{beck_test_2002}.
 \item[Wiederholbarkeit (Repeatability] Ein Test wird als wiederholbar bezeichnet, wenn er mehrmals hintereinander ausgeführt werden kann, und dabei jedes mal dasselbe Ergebnis liefert. Problematisch sind dabei z.B. Datum und Zeit, sowie Zufallsfunktionen
 \item[Klarheit (Clarity)] Ein Test ist klar, wenn sein Zweck sofort verständlich wird. Damit wird einerseits die Lesbarkeit gemeint. Anderseits schließt dies auch ein, ob der Test genau eine Eigenschaft testet und nicht redundant zu anderen Tests ist. Dies hat zur Folge, dass die Tests wartbarer werden und als Code Dokumenation dienen können.
 \item[Präzise (Conciseness)] Ein Test ist präzise, wenn er so wenig Code und so wenige Objekte wie möglich benötigt, um sein Ziel zu erreichen. Eine Auswirkung ist, dass der Test schneller wird.
 \item[Robustheit (Robustness)] Ein Test ist robust, wenn es eine direkte Korrelation zum zu testenden Code gibt: Ist der Code korrekt, so ist der Test erfolgreich. Ist der Code falsch, so schlägt der Test fehl. Nicht-robuste Tests werden auch "`zerbrechlich"' (brittle) genannt. Dazu zählen auch sogenannte tautologische Tests, die immer erfolgreich Verlaufen, und keine Aussage über den zugrunde liegenden Programmcode geben
 \end{description}

% Einige dieser Punkte können durch Metriken überprüft werden. Dazu mehr im Abschnitt \ref{sec:metrics}.
%  Code Craft S 144