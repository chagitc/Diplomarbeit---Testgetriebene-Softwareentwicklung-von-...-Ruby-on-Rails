\section{Softwaretests}

\subsection{Warum testen}

(Softwarefehler) \url{http://www.nfranze.de/download/Diplomarbeit_Nico_Franze.pdf}
  Zusammenfassende Begriffsdefinition

\subsection{Arten von Tests}

\subsection{Testtools in Ruby}
\subsubsection{Test::Unit und Minitest}
Test::Unit (Ruby 1.8.7) und Minitest (1.9.2) sind die Testbibliotheken, die Ruby standardmäßig mitbringt. Beide basieren auf dem xUnit, bzw. SUnit Design von Kent Beck, und sind für Nutzer von JUnit oder NUnit leicht nachvollziehbar.

Für eine zu testende Klasse wird eine analoge Testklasse erstellt. Diese trägt per Definition denselben Namen wie die zu testende Klasse mit einem "`test"' am Anfang. Um z.B. eine Klasse "`job"' zu testen, wird eine Datei \texttt{test\_job.rb} (Ruby Standard) oder \texttt{job\_test.rb} (Rails Standard) erstellt. Dort wiederrum wird eine Klasse mit Namen \texttt{TestJob} definiert. 

Ein Beispieltest sieht z.B. so aus:
\begin{lstlisting}
require "job"

class TestJob < Test::Unit::TestCase
  def setup
    @job = Job.create
  end
  
  def teardown
    Job.delete_all
  end
  
  def test_job_exists
    @job.title =  "Ruby on Rails Entwickler
    @job.add_location_to_title( "Dresden")
    
    assert_equal( "Ruby on Rails Entwickler in Dresden",  Job.first.title)
  end
end
\end{lstlisting}
Unsere Klasse TestJob erbt von der TestUnit Basisklasse. Sie beinhaltet die Methoden "`setup"' und "`teardown"', die jeweils vor, respektive nach jedem einzelnen Testfall aufgerufen werden.
In der Setup-Methode nehmen wir z.B. das Anlegen eines Jobs vor, in der Teardown Methode löschen wir alle Jobs in der Datenbank, um einen sauberen Test zu gewährleisten

Danach können nun beliebig viele Testmethoden folgen, deren Namen mit \texttt{test\_} beginnen müssen.
Jede Testmethode besteht in der Regel aus einer Initialisierung (kann in die setup-Methode ausgelagert werden), der Ausführung einer zu testenden Aktion und dem Prüfen der danach geltenden Eigenschaften mittels Assertions. Diese Zusicherungen sind Prädikate die oft Gleichheit oder Boolsche Rückgabewerte prüfen.

%TODO Weitere Prädikate erwähnen

Rails erledigt das Anlegen und Löschen von Testdaten selbständig. Diese werden als Fixtures bezeichnet und extern definiert. Alternativ ist der Einsatz sogenannter Factories möglich, um schnell Objekte mit bestimmten Eigenschaften zu erstellen. In jedem Fall setzt Rails die Datenbank nach jedem einzelnen Test zurück.


\subsubsection{Rspec}
\subsubsection{Cucumber}